% !TeX TXS-program:compile = txs:///lualatex

\documentclass[a4paper,11pt]{article}
\usepackage[revgoku]{cp-base}
\graphicspath{{./graphics/}}
%variables
\donnees[typedoc=CHAP,numdoc=09,classe=1\up{ère} 2M2,matiere={SPÉ MATHS},mois=Mars,annee=2022,titre=Suites (v2)]
%formatage
\author{Pierquet}
\title{\nomfichier}
\hypersetup{pdfauthor={Pierquet},pdftitle={\nomfichier},allbordercolors=white,pdfborder=0 0 0,pdfstartview=FitH}
%divers
\lhead{\entete{\matiere}}
\chead{\entete{\lycee}}
\rhead{\entete{\classe{} - \mois{} \annee}}
\cfoot{\logolycee{}}
\rfoot{\pied{\numeropagetot}}
\urlstyle{same}

\begin{document}

\pagestyle{fancy}

\part{CH09 - Suites arithmétiques, géométriques - Exercices (Correction)}

\medskip

\exonum{0}

\begin{enumerate}
	\item 
	\begin{enumerate}
		\item On a directement le fait que $u_{n+1}=u_n + r = u_n + 6$ pour tout entier $n$.
		\item Ainsi on a $u_n = u_0 + nr = 5 + 6n$ pour tout entier $n$.
		\item De ce fait, $u_7=u_0+7r=5+6\times7=47$ et $u_{50}=u_0+50r=5+6\times50=305$.
		\item La suite $\suiten$ est arithmétique de raison $r=6>0$, elle est donc monotone croissante.
		\item On a $u_0 + u_1 + \ldots + u_{20} = 21 \times \dfrac{u_0+u_{20}}{2}=21 \times \dfrac{5 + (5 + 6 \times 20)}{2} = 21 \times \dfrac{130}{2}=\num{1365}$.
	\end{enumerate}
	\item 
	\begin{enumerate}
		\item On a directement le fait que $v_{n+1}=v_n \times q = 1,05v_n$ pour tout entier $n \pg 1$.
		\item Ainsi on a $v_n = v_1 \times q^{n-1} = \num{1000} \times 1,05^{n-1}$ pour tout entier $n \pg 1$.
		\item De ce fait, $v_7=v_1 \times q^6 = \num{1000} \times 1,05^6 \approx \num{1340,096}$ et $v_{20}=v_1 \times q^{19} = \num{1000} \times 1,05^{19} \approx \num{2526,950}$ au millième.
		\item $\suiten[v]$ est géométrique de raison $q > 1$ et de 1\up{er} terme $v_1 = \num{1000} > 0$, elle est donc monotone croissante.
		\item On a $v_1 + v_2 + \ldots + v_{13} = v_1 \times \dfrac{1-q^{13}}{1-q}=\num{1000} \times \dfrac{1-1,05^{13}}{1-1,05} \approx \num{17712,983}$ au millième.
	\end{enumerate} 
\end{enumerate}

\smallskip

\exonum{1}

\begin{enumerate}
	\item 
	\begin{enumerate}
		\item La suite $\suiten$ est arithmétique (\og on ajoute toujours $-3$ \fg) de raison $r=-3$ et de 1\up{er} terme $u_1=50$.
		\item On a ainsi $u_n = u_1 + (n-1)r=50-3(n-1)=50-3n+3=53-3n$ pour tout $n \pg 1$.
		\item La suite $\suiten$ est arithmétique de raison $r=-3<0$, elle est donc monotone décroissante.
		\item On cherche $n$ tel que $u_n <0 \ssi 53-3n < 0 \ssi 3n > 53 \ssi n > \dfrac{53}{3} \approx 17,67 \Rightarrow n \pg 18$.
	\end{enumerate}
	\item 
	\begin{enumerate}
		\item $\suiten[v]$ est géométrique (\og on multiplie toujours par $0,88$ \fg) de raison $q=0,88$ et de 1\up{er} terme $v_0=100$.
		\item On a ainsi $v_n = v_0 \times q^n = 100 \times 0,88^n$ pour tout $n \pg 1$.
		\item $\suiten[v]$ est géom. de raison $q \in \intervOO{0}{1}$ et de 1\up{er} terme $v_0 = \num{100} > 0$, elle est donc monotone décroissante.
		\item En tabulant, $\left. \begin{dcases} v_{6} \approx 46,44 \\ v_{7} \approx 40,87 \end{dcases} \right| \Rightarrow n \pg 7$.
	\end{enumerate} 
\end{enumerate}

\smallskip

\exonum{2}

\begin{enumerate}
	\item 
	\begin{enumerate}
		\item $\suiten$ étant arithmétique, on a $u_5 = u_2 + 3r \ssi 46 = 10+3r \ssi 36 = 3 r \ssi r=12$.
		\item On a, de ce fait, $u_{30}=u_2+28r = 10 + 28 \times 12 = 346$.
	\end{enumerate}
	\item 
	\begin{enumerate}
		\item $\suiten[v]$ étant géométrique, on a $v_7 = v_5 \times q^2 \ssi 6,25 = 4 \times q^2 \ssi q^2 = \dfrac{6,25}{4}=1,5625 \Rightarrow q=\pm \sqrt{1,5625}=\pm 1,25$.
		\item On a, de ce fait, $v_{11}=v_5 \times q^6 = 4 \times (\pm 1,25)^6 \approx 15,26$ au centième.
	\end{enumerate}
\end{enumerate}

\smallskip

\exonum{1}

\begin{enumerate}
	\item Chaque mois, on augmente la somme de 2\,€.
	\begin{enumerate}
		\item On a, de ce fait, $s_2=10+2=12$ (12\,€ de côté en Février 2020) et $s_3=12+2=14$ (14\,€ en Mars 2020).
		\item Pour tout entier $n \pg 1$, $s_{n+1}=s_n+2$, donc $\suiten[s]$ est arithmétique de raison $r=2$ et de 1\up{er} terme $s_1=10$.
		\item On a ainsi $s_n = s_1 + (n-1)r=10+2(n-1)=10+2n-2=8+2n$ pour tout $n \pg 1$.
		\item La somme mise de côté en Décembre 2020 est donnée par $s_{12}=8+2\times12=32$ soit 32\,€.
		
		La somme totale mise de côté en 2020 est $S=s_1+\ldots+s_{12}=12 \times \dfrac{s_1+s_{12}}{2}=12 \times \dfrac{10+32}{2}=252$ soit 252\,€.
	\end{enumerate}
	\item Chaque seconde, on augmente le nombre de bactéries de 25\,\%, donc on multiplie par $CM=1+\tfrac{25}{100}=1,25$.
	\begin{enumerate}
		\item On a $b_0=5$. Puis $b_1=5 \times 1,25 = 6,25$ (au bout de 1~s) et $b_2=6,25 \times 1,25 = 7,8125$ (au bout de 2~s).
		\item Pour tout entier $n$, $b_{n+1}=b_n \times 1,25$, donc $\suiten[b]$ est géométrique de raison $q=1,25$ et de 1\up{er} terme $b_0=5$.
		\item On a ainsi $b_n = b_0 \times q^{n} =  5 \times 1,25 ^n$ pour tout $n$.
		\item Au bout d'une minute, on calcule $b_{60}=5 \times 1,25^{60} \approx \num{3262652}$ soit \num{3262652} bactéries !
		\item En tabulant, $\left. \begin{dcases} b_{20} \approx 433,68 \\ b_{21} \approx 542,10 \end{dcases} \right| \Rightarrow n \pg 21$, donc il y aura plus de 500 bactéries au bout de 21 secondes.
	\end{enumerate}
\end{enumerate}

\smallskip

\exonum{3}

\begin{enumerate}
	\item 
	\begin{enumerate}
		\item On a $u_1=0,8 \times u_0 + 18 = 0,8 \times 65 + 18=70$ et $u_2=0,8 \times u_1 + 18 = 0,8 \times 70 + 18=74$.
		\item 
		\begin{itemize}[leftmargin=*]
			\item $\begin{dcases} u_1 - u_0 = 70-65=5 \\ u_2-u_1=74-70=4 \end{dcases}$, la suite $\suiten$ ne peut donc pas être arithmétique ;
			\item $\begin{dcases} \tfrac{u_1}{u_0} = \tfrac{70}{65} \approx 1,077 \\ \tfrac{u_2}{u_1} = \tfrac{74}{70} \approx 1,057 \end{dcases}$, la suite $\suiten$ ne peut donc pas être géométrique.
		\end{itemize}
	\end{enumerate}
	\item 
	\begin{enumerate}
		\item On a $v_0=u_0-90=65-90=-25$, $v_1=u_1-90=70-90=-20$ et $v_2=u_2-90=74-90=-16$.
		
		%\textbf{NB} : la suite $\suiten[v]$ semble être géométrique de raison $\tfrac{-20}{-25} = \tfrac{-16}{-20} = 0,8$.
		\item On a $v_{n+1} = u_{n+1} - 90 = (0,8u_n+18)-90=0,8u_n-72$ pour tout entier $n$.
		\item Et, $v_{n+1}= 0,8 \left(u_n - \dfrac{72}{0,8}\right) = 0,8 \big( u_n - 90 \big) = 0,8v_n$ pour tout $n$.
		
		On en déduit donc que $\suiten[v]$ est géométrique de raison $q=0,8$ et de 1\up{er} terme $v_0=-25$. 
		\item On a ainsi $v_n = v_0 \times q^n = -25 \times 0,8 ^n$ pour tout $n$.
	\end{enumerate}
	\item Sachant que $v_n=u_n-90$ pour tout $n$, on a $u_n=v_n+90=-25 \times 0,8 ^n+90$ pour tout $n$.
	\item 
	\begin{enumerate}
		\item On a ainsi $u_{10}=-25 \times 0,8^{10}+90 \approx 87,32$ au centième.
		\item En utilisant la formule explicite $u_n=-25 \times 0,8 ^n+90$ pour tout $n$ :
		
		$\left. \begin{dcases} \text{la \og partie \fg~} 0,8^n \text{ est monotone décroissante car } 0< 0,8 < 1 \\ \text{la \og partie \fg~} -25 \times \text{ est négative} \\ \text{la \og partie \fg~} +90 \text{ ne change rien} \end{dcases} \right| \Rightarrow$ la suite $\suiten$ est monotone croissante.
		\item En tabulant, $\left. \begin{dcases} u_{7} \approx 84,73 \\ u_{8} \approx 85,81 \end{dcases} \right| \Rightarrow n \pg 8$.
	\end{enumerate}
\end{enumerate}

\smallskip

\exonum{5}

\smallskip

\begin{enumerate}
	\item En étant attentif à la concordance des indices :
	
	$u_2=\dfrac{1+1}{3\times1} \times u_1 = \dfrac23 \times \dfrac13=\dfrac29$ puis $u_3=\dfrac{2+1}{3\times2} \times u_2 = \dfrac36 \times \dfrac29=\dfrac19$ puis $u_4=\dfrac{3+1}{3\times3} \times u_3 = \dfrac49 \times \dfrac19=\dfrac{4}{81}$.
	\item 
	\begin{enumerate}
		\item On a $v_{n+1}=\dfrac{u_{n+1}}{n+1}= \dfrac{\cancel{n+1}}{3n}u_n \times \dfrac{1}{\cancel{n+1}}=\dfrac{u_n}{3n}=\dfrac13 \times \dfrac{u_n}{n} = \dfrac13 \times v_n$ pour tout $n \pg 1$.
		\item La suite $\suiten[v]$ est donc géométrique de raison $q=\dfrac13$ et de 1\up{er} terme $v_1=\dfrac{u_1}{1}=\dfrac13$.
		
		On en déduit donc que $v_n = v_1 \times q^{n-1}=\dfrac13 \times \left( \dfrac13 \right)^{n-1} = \left( \dfrac13 \right)^{n}$ pour tout $n \pg 1$.
	\end{enumerate}
	\item Ainsi, de $v_n=\dfrac{u_n}{n}$, on déduit que $u_n = n \times v_n = n \times \left( \dfrac13 \right)^{n}$ pour tout $n \pg 1$.
	\item 
	\begin{enumerate}
		\item On a $u_{n+1}-u_n = (n+1) \left( \dfrac13\right)^{n+1} - n \left( \dfrac13\right)^{n} = \left( \dfrac13\right)^{n+1} \times \big[ n+1 - 3 \times n\big] = (1-2n) \left( \dfrac13\right)^{n+1}$ pour tout entier $n \pg 1$.
		\item On a $(1-2n) <0$ et $\left( \dfrac13\right)^{n+1} >0$, donc par produit, $u_{n+1}-u_n < 0$, donc $\suiten$ est monotone décroissante.
	\end{enumerate}
	\item En tabulant, $\left. \begin{dcases} u_{15} \approx 1,04 \times 10^{-6} \\ u_{16} \approx 3,71 \times 10^{-17} \end{dcases} \right| \Rightarrow n \pg 16$.
\end{enumerate}

\end{document}