% !TeX TXS-program:compile = txs:///arara
% arara: lualatex: {shell: no, synctex: yes, interaction: batchmode}
% arara: pythontex: {rerun: modified} if found('pytxcode', 'PYTHONTEX#py')
% arara: lualatex: {shell: no, synctex: yes, interaction: batchmode} if found('pytxcode', 'PYTHONTEX#py')
% arara: lualatex: {shell: no, synctex: yes, interaction: batchmode} if found('log', '(undefined references|Please rerun|Rerun to get)')

\documentclass[a4paper,11pt]{article}
\usepackage[revgoku]{cp-base} %avec options possibles parmi breakable (tcbox), sujetl (exos),  (pour faire "comme avant"), etc...
\graphicspath{{./graphics/}}
%variables
\donnees[classe=1\up{ère} 2M2,matiere={[SPÉ.MATHS]},typedoc=TD,numdoc=5,titre={Étude de salaires},mois=Mars,annee=2022]

%formatage
\author{Pierquet}
\title{\nomfichier}
\hypersetup{pdfauthor={Pierquet},pdftitle={\nomfichier},allbordercolors=white,pdfborder=0 0 0,pdfstartview=FitH}
%divers
\lhead{\entete{\matiere}}
\chead{\entete{\lycee}}
\rhead{\entete{\classe{} - \mois{} \annee}}
%\rhead{\entete{\classe{} - Chapitre }}
\lfoot{\pied{\matiere}}
\cfoot{\logolycee{}}
\rfoot{\pied{\numeropagetot}}

\begin{document}

\pagestyle{fancy}

\part{TD05 - Étude de salaires}

\smallskip

\nomprenomtcbox

\smallskip

Camille et Dominique ont été embauchés au même moment dans une entreprise et ont négocié leur contrat :

\begin{itemize}%[label=\scriptsize\faIcon{money-bill-alt}]
	\item[\scriptsize\faIcon{money-bill-alt}] Camille a commencé en 2018 avec un salaire annuel de \num{14400}\,€ ;
	\item[\scriptsize\faIcon{money-bill-alt}] le salaire de Dominique était, cette même année, de \num{13200}\,€ ;
	\item[\scriptsize\faChartLine] le salaire de Camille augmente de 600\,€ par an ;
	\item[\scriptsize\faChartLine] celui de Dominique augmente de 4\,\% par an. 
\end{itemize} 
%
\begin{enumerate}
	\item Quels étaient les salaires annuels de Camille et de Dominique en 2019 ? En 2020 ? Justifier les réponses.
	\item \textbf{Étude du salaire de Camille.} On note $c_n$ le salaire de Camille en l’année $2018+n$. On a donc $c_0 = \num{14400}$.
	\begin{enumerate}
		\item Quelle est la nature de la suite $\suiten[c]$ ? Justifier la réponse.
		\item Déterminer l'expression de $c_n$ en fonction de $n$. 
		\item Calculer le salaire de Camille en 2028.
		\item Déterminer, en détaillant la méthode, en quelle année le salaire de Camille dépassera \num{25000}\,€. 
	\end{enumerate}
	\item \textbf{Étude du salaire de Dominique.} On note $d_n$ le salaire de Dominique en l’année $2018+n$. 
	\begin{enumerate}
		\item Préciser la valeur de $d_0$.
		\item Exprimer $d_{n+1}$ en fonction de $d_n$ puis en déduire la nature de la suite $\suiten[d]$.
		\item Déterminer l'expression de $d_n$ en fonction de $n$.
		\item Calculer le salaire de Dominique en 2028. On arrondira le résultat à l’euro. 
	\end{enumerate}
	\item \textbf{Évolution des deux salaires.} On veut déterminer à partir de quelle année le salaire de Dominique dépassera celui de Camille. Pour cela, on dispose du programme incomplet ci-dessous écrit 
	en langage \calgpython.
	\begin{enumerate}
		\item Compléter les quatre parties en pointillé du programme ci-dessous : 
		\begin{envpython}[12cm]
			def algo() : 
				C = 14400
				D = 13200
				n = 0
				while .................. :
					C = ................
					D = ................
					n = ................
				return n
		\end{envpython}
		\item Déterminer, en détaillant, en quelle année le salaire de Dominique dépassera celui de Camille.
	\end{enumerate}
	\item \textbf{Salaires cumulés entre 2018 et 2028.}
	\begin{enumerate}
		\item Calculer la somme $\sum_{i=0}^{10} d_i = d_0+d_1+\ldots+d_{10}$ et interpréter le résultat.
		\item Déterminer, en détaillant, le montant total des salaires perçus par Camille entre 2018 et 2028.
		\item Compléter les algorithmes suivants, écrits en \calgpython, afin qu'ils permettent de retrouver le résultat des questions \ptno{5}\pta{a} et \ptno{5}\pta{b}.
	\end{enumerate}
	\begin{minipage}{0.45\textwidth}
		\begin{envpython}[8.5cm]
			def cumul_camille() : 
				C = 14400
				S = 14400 
				for i in range(10):
					C = ................
					S = S + ............
				return ...
		\end{envpython}
	\end{minipage}\hfill
	\begin{minipage}{0.45\textwidth}
		\begin{envpython}[8.5cm]
			def cumul_dominique() : 
				D = .....
				S = 13200 
				for i in range(10):
					D = ................
					S = S + ............
				return ...
		\end{envpython}
	\end{minipage}
\end{enumerate}

\end{document}