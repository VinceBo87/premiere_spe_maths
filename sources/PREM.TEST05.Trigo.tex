% !TeX TXS-program:compile = txs:///lualatex

\documentclass[a4paper,11pt]{article}
\usepackage[revgoku]{cp-base}
\graphicspath{{./graphics/}}
%variables.
\donnees[classe={1\up{ère} 2M2},matiere={[SPÉ.MATHS]},mois={Jeudi 27 Janvier},annee=2022,typedoc=TEST~,numdoc=5,duree={15 minutes}]
%formatage
\author{Pierquet}
\title{\nomfichier}
\hypersetup{pdfauthor={Pierquet},pdftitle={\nomfichier},allbordercolors=white,pdfstartview=FitH}
%divers
\lhead{\entete{Durée : \duree}}
\chead{\entete{\lycee}}
\rhead{\entete{\classe{} - \mois{} \annee}}
\lfoot{\pied{\matiere}}
\cfoot{\logolycee{}}
\fancypagestyle{sujetA}{\fancyhead[R]{\entete{\classe{}A - \mois{} \annee}}}
\fancypagestyle{sujetB}{\fancyhead[R]{\entete{\classe{}B - \mois{} \annee}}}
\usepackage{tabularray}
\newcommand\cercletrigo{%
	\draw (0,0) circle[radius=2] ;
	\draw (-2.25,0) -- (2.25,0) ;
	\draw (0,-2.25) -- (0,2.25) ;
	\draw[->,>=stealth'] (0,0) -- (2,0) ;
	\draw[->,>=stealth'] (0,0) -- (0,2) ;
	%\draw (0,0) node[below left] {$O$} ;
}

\begin{document}

\pagestyle{fancy}

\part{TEST05 - Trigonométrie}%SUJETA

\setcounter{numexos}{0}

\medskip

\nomprenomtcbox

\medskip

\exonum{}

\medskip

À l'aide des informations présentes sur le cercle trigonométrique suivant, déterminer un angle, en radian, associé à chacun des points M, N et P.

\medskip

\begin{minipage}{3.25cm}
	\begin{tikzpicture}[x=0.75cm,y=0.75cm,thick]
		\cercletrigo
		\draw[densely dotted] (-60:2)--(60:2) (0,0)--(-135:2) (-1,0)--(-120:2) (0,-1)--(-150:2);
		\filldraw (60:2) circle[radius=2pt] node[font=\sffamily,above right=0pt] {$M$} (-135:2) circle[radius=2pt]  node[font=\sffamily,below left=0pt] {$N$} (-180:2) circle[radius=2pt]  node[font=\sffamily,above left=0pt] {$P$} ;
		\draw (0.5,0) node[font=\scriptsize] {/\!/} (1.5,0) node[font=\scriptsize] {/\!/} ;
	\end{tikzpicture}
\end{minipage}\hfill~
\begin{minipage}{14cm}
	\papierseyes*{17}{4}
\end{minipage}

\medskip

\exonum{}

\medskip

Pour chacun des angles suivants, donner sa mesure principale, son cosinus ainsi que son sinus :

\begin{center}
	\begin{tblr}{hlines,vlines,width=0.85\linewidth,colspec={X[c]X[c]X[c]X[c]},row{1}={font=\sffamily\bfseries},row{2-Z}={1.05cm}}
	angle & mesure principale & cosinus & sinus \\
	$\dfrac{17\pi}{4}$ & & & \\
	$\dfrac{-35\pi}{2}$ & & & \\
	$\dfrac{112\pi}{3}$ & & & \\
	$\dfrac{-5\pi}{6}$ & & & \\
\end{tblr}
\end{center}

\medskip

\exonum{}

\medskip

Résoudre les équations suivantes :

\begin{enumerate}
	\item $\cos(x)=-\dfrac12$ ;
\end{enumerate}
\papierseyes{22}{2}
\begin{enumerate}[resume]
	\item $\sin(x)=\dfrac{\sqrt{2}}{2}$ ;
\end{enumerate}
\papierseyes{22}{2}
\begin{enumerate}[resume]
	\item $3\sin(x)-3=0$.
\end{enumerate}
\papierseyes{22}{3}


\end{document}