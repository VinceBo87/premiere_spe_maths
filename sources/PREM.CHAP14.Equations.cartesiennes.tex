% !TeX TXS-program:compile = txs:///arara
% arara: lualatex: {shell: no, synctex: yes, interaction: batchmode}
% arara: pythontex: {rerun: modified} if found('pytxcode', 'PYTHONTEX#py')
% arara: lualatex: {shell: no, synctex: yes, interaction: batchmode} if found('pytxcode', 'PYTHONTEX#py')
% arara: lualatex: {shell: no, synctex: yes, interaction: batchmode} if found('log', '(undefined references|Please rerun|Rerun to get)')

\documentclass[a4paper,11pt]{article}
\usepackage[]{cp-base}
\graphicspath{{./graphics/}}
%variables
\donnees[%
classe=1\up{ère} 2M2,
matiere={[SPÉ.MATHS]},
typedoc=CHAPITRE~,
numdoc=14,
titre={Applications du produit scalaire}
]

%formatage
\author{Pierquet}
\title{\nomfichier}
\hypersetup{pdfauthor={Pierquet},pdftitle={\nomfichier},allbordercolors=white,pdfborder=0 0 0,pdfstartview=FitH}
%divers
\lhead{\entete{\matiere}}
\chead{\entete{\lycee}}
\rhead{\entete{\classe{} - Chapitre \thepart}}
\lfoot{\pied{\matiere}}
\cfoot{\logolycee{}}
\rfoot{\pied{\numeropagetot}}

\begin{document}

\pagestyle{fancy}

\newcommand{\coord}[3]{\vect{#1}\begin{pmatrix}#2\\#3\end{pmatrix}}

\part{CH14 - Compléments sur le produit scalaire}

\section{Applications du produit scalaire}

\subsection{Théorème d'Al-Kashi}

\begin{cthm}
Dans un triangle $ABC$, on pose $AB=c$, $AC=b$ et $BC=a$ alors :
\begin{itemize}
	\item $a^2=b^2+c^2-2bc \cos(\widehat{A})$.
	\item $b^2=a^2+c^2-2ac \cos(\widehat{B})$.
	\item $c^2=a^2+b^2-2ab \cos(\widehat{C})$.
\end{itemize}
\begin{center}
	\begin{tikzpicture}[x=0.75cm,y=0.75cm,line join=bevel]
		\tkzDefPoints{3/1/A,8/1/B,5/5/C}
		\tkzLabelAngle[ForestGreen!50!black,pos=1.1](C,B,A){$\widehat{B}$}
		\tkzLabelAngle[ForestGreen!50!black,pos=1.1](A,C,B){$\widehat{C}$}
		\tkzLabelAngle[ForestGreen!50!black,pos=1.1](B,A,C){$\widehat{A}$}
		\tkzMarkAngles[mark=none,thick,size=0.75](C,B,A B,A,C A,C,B)
		\tkzFillAngles[fill=ForestGreen,size=0.75,opacity=.75](C,B,A B,A,C A,C,B)
		\tkzLabelSegment[swap](B,C){$a$}
		\tkzLabelSegment[](A,C){$b$}
		\tkzLabelSegment[swap](A,B){$c$}
		\tkzDrawPolygon[thick](A,B,C)
		\tkzLabelPoints[above](C)
		\tkzLabelPoints[below left](A)
		\tkzLabelPoints[below right](B)
	\end{tikzpicture}
\end{center}
\end{cthm}

\begin{chistoire}
\vspace{-0.22cm}
\lettrine[findent=.5em,nindent=0pt,lines=3,image,novskip=0pt]{alkashi}{}À Samarkand, le savant perse Jemshid ibn Massoud al Kashi (1380--1430) vit sous la protection du prince Ulugh-Beg (1394--1449) qui a fondé une Université comprenant une soixantaine de scientifiques qui étudient la théologie et les 
sciences.

Dans son Traité sur le cercle (1424), al Kashi calcule le rapport de la circonférence à son rayon pour obtenir une valeur approchée de $2\pi$ avec une précision jamais atteinte. Il obtient 9 positions exactes en base 60 soit 16 décimales exactes : $2\pi \approx \num{6,2831853071795865}$.
\end{chistoire}

\begin{cdemo}[ n°1]
On a, par exemple, $a^2=BC^2$.

Et donc $a^2=BC^2=\vect{BC}^2=\left(\vect{BA}+\vect{AC}\right)^2=\left(\vect{AC}-\vect{AB}\right)^2=\vect{AC}^2+\vect{AB}^2-2\vect{AC}\cdot\vect{AB}$.

Or $\vect{AC}\cdot\vect{AB}=AC \times AB \times \cos\big(\widehat{A}\big) = bc\,\cos\big(\widehat{A}\big)$.

D'où $a^2=b^2+c^2-2bc\,\cos\big(\widehat{A}\big)$.
\end{cdemo}

\begin{cdemo}[ n°2]
On a, par exemple, $\vect{AB}\cdot\vect{AC}=AB \times AC \times \cos\big(\widehat{A}\big)=bc\,\cos\big(\widehat{A}\big)$.

Et $\vect{AB}\cdot\vect{AC}=\dfrac{1}{2} \left( AB^2+AC^2-BC^2 \right)=\dfrac{1}{2} \left( b^2+c^2-a^2 \right)$.

Ainsi $b^2+c^2-a^2=2bc\,\cos\big(\widehat{A}\big) \ssi a^2=b^2+c^2-2bc\,\cos\big(\widehat{A}\big)$.
\end{cdemo}

\begin{cvoc}
\og Résoudre un triangle \fg{}, c'est déterminer tous les côtés et les angles du triangle.
\end{cvoc}

\begin{crmq}
Le théorème d'Al-Kashi est également nommé \og théorème de Pythagore généralisé \fg{}, car en effet le théorème de Pythagore est un cas particulier du théorème d'Al-Kashi avec un angle droit (de $\cos=0$).
\end{crmq}

\subsection{Étude d'un ensemble de points}

\begin{cthm}[ de la médiane]
Soit $ABC$ un triangle et $I$ est le milieu de $[AB]$, on a $\vect{CA} \cdot \vect{CB} =CI^2- \dfrac{AB^2}{4}$.
\begin{center}
	\begin{tikzpicture}[x=0.75cm,y=0.75cm,line join=bevel]
		\tkzDefPoints{3/0/A,8/1/B,4/4/C}
		\tkzDefMidPoint(A,B) \tkzGetPoint{I}
		\tkzDrawPolygon[thick](A,B,C)
		\tkzDrawSegments[thick,densely dashed](C,I)
		\tkzLabelPoints[above](C)
		\tkzLabelPoints[below](I)
		\tkzLabelPoints[below left](A)
		\tkzLabelPoints[below right](B)
		\tkzMarkSegments[mark=s||,size=4pt](A,I I,B)
	\end{tikzpicture}
\end{center}
\end{cthm}

\begin{cdemo}
On a $\vect{CA} \cdot \vect{CB} = \left(\vect{CI}+\vect{IA}\right)\cdot\left(\vect{CI}+\vect{IB}\right)=\left(\vect{CI}+\vect{IA}\right)\cdot\left(\vect{CI}-\vect{IA}\right)=CI^2-IA^2=CI^2-\left(\dfrac{AB}{2}\right)^2=CI^2-\dfrac{AB^2}{4}$.
\end{cdemo}

\subsection{Caractérisation d'un cercle}

\begin{cprop}
Un point $M$ appartient au cercle $\mathscr{C}$ de diamètre $[AB]$ si et seulement si $\vect{MA}\cdot \vect{MB}=0$.
\begin{center}
	\begin{tikzpicture}[scale=0.75,line join=bevel]
		\tkzDefPoints{3/0/A,8/1/B}
		\tkzDefMidPoint(A,B)\tkzGetPoint{I}
		%\tkzDefCircle[diameter](A,B)\tkzGetPoint{O}
		\tkzDefPointBy[rotation=center I angle 50](B)
		\tkzGetPoint{M}
		\tkzMarkRightAngle[thick,fill=ForestGreen!50,size=.3](A,M,B)
		\tkzDrawPoint(I)
		\tkzDrawCircle[thick](I,B)
		%\tkzDrawCircle[diameter,thick](A,B)
		\tkzDrawSegments[thick](A,B M,A M,B)
		\tkzLabelPoints[left](A)
		\tkzLabelPoints[below](I)
		\tkzLabelPoints[right](B)
		\tkzLabelPoints[above](M)
	\end{tikzpicture}
\end{center}
\end{cprop}

\begin{cdemo}
Soit $\mathscr{C}$ le cercle de diamètre $[AB]$.

Le cercle de diamètre $[AB]$, privé de $A$ et de $B$, est l'ensemble des points $M$ du plan tels que le triangle $MAB$ est rectangle en $M$, c'est à dire l'ensemble des points $M$ tels que : $\vect{MA}\cdot\vect{MB}=0$ avec $\vect{MA}\neq0$ et $\vect{MB}\neq0$.

De plus, si $M=A$ ou si $M=B$, on a $\vect{MA}\cdot\vect{MB}=0$.
\end{cdemo}

\begin{crmq}
Au collège, cette propriété est connue comme celle d'un \og triangle inscrit dans un cercle de diamètre l'un des côtés \fg. C'est également une conséquence de \og l'angle au centre \fg.
\end{crmq}

\pagebreak

\section{Équation cartésienne de droites}

\subsection{Vecteur directeur}

\begin{cdefi}
Soit $\mathscr{D}$ une droite. Un vecteur non nul $\vect{d}$ est dit \textbf{directeur} de la droite $\mathscr{D}$ s'il a la même direction que la droite.

Il peut être  \og sur \fg{} la droite, c'est-à-dire avec une origine et une extrémité qui sont des points de la droite, ou sur une droite parallèle. Ci-dessous, $\vect{d}$ et $\vect{u}$ dirigent $\mathscr{D}$ mais pas $\vect{w}$.


\begin{center}
	\definecolor{qqccqq}{rgb}{0.,0.8,0.}
	\definecolor{ffqqqq}{rgb}{1.,0.,0.}
	\definecolor{qqqqff}{rgb}{0.,0.,1.}
	\begin{tikzpicture}[line cap=round,line join=round,>=latex,x=0.8cm,y=0.8cm]
		\begin{axis}[x=0.5cm,y=0.5cm,axis lines=middle,%
			ymajorgrids=true,xmajorgrids=true,xmin=-1.1,xmax=7.1,%
			ymin=-4.1,ymax=6.1,xtick={0,2,4,6},ytick={-4,-2,2,4}]
			\clip(-1.1,-4.1) rectangle (7.1,6.1);
			\draw (-0.35,-0.35) node  {$0$};
			\draw [line width=1.pt,color=qqqqff,domain=-1.1:7.1] plot(\x,{(-1.+2.*\x)/1.});
			\draw [color=qqqqff](3.4,5.56) node[anchor=north west] {$\mathscr{D}$};
			\draw [->,line width=2.pt,color=ffqqqq] (1.,1.) -- (2.,3.);
			\draw [color=ffqqqq](1.52,2.16) node[anchor=north west] {$\vect{d}$};
			\draw [color=ffqqqq](1.7,-1.54) node[anchor=north west] {$\vect{u}$};
			\draw [color=qqccqq](5.62,2.82) node[anchor=north west] {$\vect{w}$};
			\draw [->,line width=2.pt,color=ffqqqq] (3,1.) --(1.,-3.) ;
			\draw [->,line width=2.pt,color=qqccqq] (5.,3.) -- (6.,1.);
		\end{axis}
	\end{tikzpicture}
\end{center}
\end{cdefi}

\begin{crmq}
Pour deux droites, on dit qu'elles sont \emph{parallèles}, deux vecteurs seront dits plutôt \emph{colinéaires}, dans le cas d'un vecteur et d'une droite, on dira que le vecteur \emph{dirige} la droite ou en est un \emph{vecteur directeur}.
\end{crmq}

\begin{cprop}
Si $\vect{d}$ dirige $\mathscr{D}$, alors tout vecteur non nul \textbf{colinéaire} à $\vect{d}$ est aussi un vecteur directeur de $\mathscr{D}$.
\end{cprop}

\begin{cprop}
Une droite de coefficient directeur $m$ est dirigée par le vecteur de coordonnées  $\begin{pmatrix}  1\\ m \end{pmatrix}$.
\end{cprop}

\subsection{Vecteur normal}

\begin{cdefi}
Soit $\mathscr{D}$ une droite de vecteur directeur $\vect{d}$. Un vecteur non nul $\vect{n}$ est dit \textbf{normal} à la droite $\mathscr{D}$ si $\vect{n} \cdot \vect{d}=0$.

Ci-dessous, $\vect{n}$ est un vecteur normal à $\mathscr{D}$. Sa direction est \emph{perpendiculaire} à celle de $\mathscr{D}$.
\begin{center}	
	\definecolor{qqccqq}{rgb}{0.,0.8,0.}
	\definecolor{ffqqqq}{rgb}{1.,0.,0.}
	\definecolor{qqqqff}{rgb}{0.,0.,1.}
	\begin{tikzpicture}[line cap=round,line join=round,>=latex,x=1.0cm,y=1.0cm]
		\begin{axis}[x=0.5cm,y=0.5cm,axis lines=middle,ymajorgrids=true,xmajorgrids=true,%
			xmin=-1.1,xmax=7.1,ymin=-4.1,ymax=6.1,xtick={},ytick={4,-2,2,4}]
			\clip(-1.1,-4.1) rectangle (7.1,6.1);
			\draw (-0.35,-0.35) node  {$0$};
			\draw[line width=1.pt] (0.9794733192202056,0.010263340389897152) -- (1.1692099788303085,0.38973665961010284) -- (0.7897366596101029,0.5794733192202057) -- (0.6,0.2) -- cycle;
			\draw [line width=1.pt,color=qqqqff,domain=-1.1:7.1] plot(\x,{(-1.+2.*\x)/1.});
			\draw [color=qqqqff](3.4,5.56) node[anchor=north west] {$\mathscr{D}$};
			\draw [->,line width=2.pt,color=ffqqqq] (1.,1.) -- (2.,3.);
			\draw [color=ffqqqq](1.52,2.16) node[anchor=north west] {$\vect{d}$};
			\draw [color=qqccqq](4,-1) node[anchor=north west] {$\vect{n}$};
			\draw [->,line width=2.pt,color=qqccqq] (3.,-1.) -- (6.,-2.5);
			\draw [line width=1.pt,dotted,domain=-1.1:6.1] plot(\x,{(--1.-1.*\x)/2.});
		\end{axis}
	\end{tikzpicture}
\end{center}
\end{cdefi}

\begin{cmethode}
Pour définir la \textit{direction} d'une droite, on peut soit en donner un vecteur directeur, soit un vecteur normal, suivant les situations.
\end{cmethode}

\begin{cprop}[s]
\vspace{-0.2cm}
\begin{itemize}[leftmargin=*]
	\item Si $\vect{n}$ est un vecteur normal à $\mathscr{D}$, alors tout vecteur non nul \textbf{colinéaire} à $\vect{n}$ est également un vecteur normal de $\mathscr{D}$ ;
	\item Tout vecteur normal à $\mathscr{D}$ est \textbf{orthogonal} à tout vecteur directeur de $\mathscr{D}$.
\end{itemize}
\end{cprop}

\subsection{Équation cartésienne de droite}

\begin{cprop}
Soit $\mathscr{D}$ une droite passant par un point $A$ et de vecteur normal $\vect{n}$.

Un point $M$ du plan appartient à la droite $\mathscr{D}$ si et seulement si $\overrightarrow{AM} \cdot \vect{n}=0$.

\smallskip

Ici, $M$ est sur la droite, mais $N$ ne l'est pas car $\vect{AN} \cdot \vect{n} \neq 0$.

\begin{center}
	\definecolor{sqsqsq}{rgb}{0.12549019607843137,0.12549019607843137,0.12549019607843137}
	\definecolor{ffqqqq}{rgb}{1.,0.,0.}
	\definecolor{qqwuqq}{rgb}{0.,0.39215686274509803,0.}
	\definecolor{qqccqq}{rgb}{0.,0.8,0.}
	\definecolor{qqqqff}{rgb}{0.,0.,1.}
	\begin{tikzpicture}[line cap=round,line join=round,>=latex,x=1.0cm,y=1.0cm]
		\begin{axis}[x=0.5cm,y=0.5cm,axis lines=middle,ymajorgrids=true,xmajorgrids=true,%
			xmin=-1,xmax=7,ymin=-1,ymax=6,xtick={2,4,6},ytick={2,4},]
			\clip(-4.08,-11.04) rectangle (21.,6.22);
			\draw (-0.35,-0.35) node  {$0$};
			\draw[line width=1.pt,color=qqwuqq,fill=qqwuqq,fill opacity=0.10000000149011612] (4.379473319220206,3.189736659610103) -- (4.189736659610102,3.569209978830308) -- (3.810263340389897,3.3794733192202058) -- (4.,3.) -- cycle; 
			\draw [line width=1.pt,color=qqqqff,domain=-4.08:21.] plot(\x,{(--1.--0.5*\x)/1.});
			\draw [color=qqqqff](6.06,5.18) node[anchor=north west] {$\mathscr{D}$};
			\draw [color=qqccqq](3.54,4.82) node[anchor=north west] {$\vect{n}$};
			\draw [->,line width=2.pt,color=qqccqq] (4.,3.) -- (3.,5.);
			\draw [->,line width=1.pt,color=ffqqqq] (2.,2.) -- (5.552,3.776);
			\draw [->,line width=1.pt] (2.,2.) -- (6.,2.);
			\begin{scriptsize}
				\draw [color=ffqqqq] (2.,2.)-- ++(-2.5pt,0 pt) -- ++(5.0pt,0 pt) ++(-2.5pt,-2.5pt) -- ++(0 pt,5.0pt);
				\draw[color=ffqqqq] (1.8,2.37) node {$A$};
				\draw [color=ffqqqq] (5.552,3.776)-- ++(-2.5pt,0 pt) -- ++(5.0pt,0 pt) ++(-2.5pt,-2.5pt) -- ++(0 pt,5.0pt);
				\draw[color=ffqqqq] (5.6,4.25) node {$M$};
				\draw [color=sqsqsq] (6.,2.)-- ++(-2.5pt,0 pt) -- ++(5.0pt,0 pt) ++(-2.5pt,-2.5pt) -- ++(0 pt,5.0pt);
				\draw[color=sqsqsq] (6.14,2.37) node {$N$};
			\end{scriptsize}
		\end{axis}
	\end{tikzpicture}
\end{center}
\end{cprop}

\begin{cthm}
Toute droite du plan admet une équation cartésienne de la forme \[ax+by+c=0\]%
où $a, b $ et $c$ sont des réels non tous nuls et où $\begin{pmatrix}  a\\ b \end{pmatrix}$ sont les coordonnées d'un vecteur \textbf{normal} à la droite.
\end{cthm}

\begin{cexemple}[ n°1]
Soit $\mathscr{D}$ est la droite passant par $A(6\,;\,4)$ et de vecteur normal $\vect{n}$ de coordonnées $\begin{pmatrix} \textcolor{ForestGreen}{-1}\\ \textcolor{orange}{2} \end{pmatrix}$ donc la droite $\mathscr{D}$ a une équation cartésienne de la forme $\textcolor{ForestGreen}{-1}x+\textcolor{orange}{2}y+c=0$.

Pour trouver la valeur de $c$, on utilise le point $A(\textcolor{red}{6\,;\,4})$ qui est sur la droite et dont les coordonnées vérifient l'équation : $\textcolor{ForestGreen}{-1}\times \textcolor{red}{6} + \textcolor{orange}{2}\times \textcolor{red}{4} +c=0 \ssi 2+c=0 \ssi c = -2$ et donc $\mathscr{D}$ a pour équation cartésienne $-x+2y-2=0$.
\end{cexemple}

\begin{cexemple}[ n°2]
$\bullet~~$Soit  $\mathscr{D'}$ : $5x-3y+8=0$.

On cherche un vecteur normal $\vect{m}$ en relevant les coefficients devant le $x$ et le $y$ dans l'équation : ici, $\vect{m}\begin{pmatrix}	5\\-3 \end{pmatrix}$.

On cherche ensuite un point $B$ de la droite en cherchant des coordonnées $(x\,;\,y)$ qui vérifient l'équation, par exemple ici $B(-1\,;\,1)$car $5 \times (-1) - 3 \times 1 +8 = 0$ (on essaie si possible de trouver des coordonnées entières, mais sinon, on peut aussi choisir n'importe quelle valeur pour l'une des deux coordonnées et calculer la seconde !).

\medskip

$\bullet~~\Delta : 2x-3=0$ a pour vecteur normal $\vect{n} \begin{pmatrix}	2\\0 \end{pmatrix}$ (donc $\Delta$ est verticale) et passe par le point de coordonnées $(1,5\,;\,0)$ . 
\end{cexemple}

\begin{cdemo}
Soit $\mathscr{D}$ la droite de vecteur normal $\vect{n} \begin{pmatrix} a\\b \end{pmatrix}$ et passant par $A(x_A\,;\,y_A)$. Si $M(x\,;\,y)$ est un point quelconque,
\begin{flalign*}
	M(x;y) \in \mathscr{D} &\ssi \vect{AM} \cdot \vect{n}=0 \\ & \ssi \begin{pmatrix} x-x_A\\y-y_A \end{pmatrix} \cdot \begin{pmatrix} a\\b \end{pmatrix} =0 \\ & \ssi a(x-x_A)+b(y-y_A)=0 \\ & \ssi ax+by+(-ax_A-by_A)=0 &
\end{flalign*}
et on retrouve bien la forme annoncée en appelant $c$ ce qui est dans la parenthèse.
\end{cdemo}

\begin{crmq}[s -- lien avec les équations réduites de droites, de la forme \boldmath$y=mx+p$\unboldmath]
\vspace{-0.2cm}
\begin{itemize}[leftmargin=*]
	\item On connaissait déjà les \textbf{équations réduites} de droites. Mais les équations réduites ne sont valables que pour les droites \textit{non verticales} (représentations de fonctions affines), tandis que \textit{toutes} les droites ont une équation cartésienne, même les droites verticales.
	\item À partir d'une équation réduite, on obtient une équation cartésienne en passant tout dans le même membre pour avoir \og  $=0$ \fg .
	\item On peut retrouver l'équation réduite $y=mx+p$ d'une droite à partir de l'équation cartésienne en isolant le $y$ d'un côté de l'égalité !
	
	Mais ceci n'est toutefois possible que s'il y a $y$ dans l'équation de la droite, autrement dit si $b \neq 0$.
	\item Une droite est verticale si et seulement si il n'y a pas de $y$ dans ses équations cartésiennes (donc si $b=0$).
	
	\item \textbf{Une} droite donnée admet une \textbf{infinité} d'équations cartésiennes, puisqu'elle admet une infinité de vecteurs normaux.
	
	Par contre, si elle admet une équation réduite, elle n'en a qu'une seule.
\end{itemize}
\end{crmq}

\section{Équation cartésienne de cercles}

\subsection{Cercle défini par centre et rayon}

\begin{cthm}
Soit $A(x_A\,;\,y_A) $ un point du plan et $r$ un réel strictement positif.

Alors le cercle $\mathscr{C}$ de centre $A$ et de rayon $r$ a pour équation cartésienne  : \[(x-x_A)^2+(y-y_A)^2=r^2.\]
\end{cthm}

\begin{cdemo}
Soit $\mathscr{C}$ le cercle de centre $A(x_A\,;\,y_A)$ et de rayon $r$. Si $M(x\,;\,y)$ est un point quelconque,
\begin{flalign*}
	M \in \mathscr{C} &\ssi AM=r \\ & \ssi \sqrt{(x-x_A)^2+(y-y_A)^2}=r \\ & \ssi (x-x_A)^2+(y-y_A)^2=r^2. &
\end{flalign*}
\end{cdemo}

\begin{cexemple}
Soit le cercle $\mathscr{C}$ ayant centre $A(2\,;\,-1)$ et pour rayon 5. Il a pour équation cartésienne  $(x-2)^2+(y+1)^2=25$.

\medskip

Le point $B(\textcolor{ForestGreen}{6}\,;\,\textcolor{ForestGreen}{2}) \in \mathscr{C}$ car $\textcolor{ForestGreen}{6}-2)^2+(\textcolor{ForestGreen}{2}+1)^2=16+9=25$.

\medskip

Mais le point $D\textcolor{red}{(-3\,;\,-2)}  \notin \mathscr{C}$ car $(\textcolor{red}{-3}-2)^2+(\textcolor{red}{-2}+1)^2=25+1=26 \neq 25$.
\end{cexemple}

\begin{crmq}
Autrement dit, l'ensemble des points $M$ du plan dont les coordonnées vérifient une égalité du type $(x-x_A)^2+(y-y_A)^2=r^2$ est l'ensemble des points du cercle de centre $A$ et de rayon $r$.

\smallskip

Il arrive aussi qu'on donne une équation cartésienne de cercle sous forme développée. Dans ce cas, il est un peu plus difficile de repérer les coordonnées du centre et le rayon : il faut alors revenir aux \textit{formes canoniques}

Par exemple,
\begin{flalign*}
	x^2+y^2-4x+5y+3=0 &\ssi \textcolor{blue}{x^2-4x }+ \textcolor{ForestGreen}{y^2+5y}=-3&\\& \ssi \textcolor{blue}{x^2-4x +2^2 }+ \textcolor{ForestGreen}{y^2+5y+2,5^2}=-3 \textcolor{blue}{+2^2} \textcolor{ForestGreen}{+2,5^2}\\ & \ssi \textcolor{blue}{(x-2)^2 }+ \textcolor{ForestGreen}{(y+2,5)^2}=-3 +4 +\frac{25}{4} \\ &\ssi (x-2)^2 + (y+2,5)^2=\frac{29}{4}  
\end{flalign*}
Donc il s'agit du cercle de centre $A(2\,;\,-2,5)$ et de rayon $\frac{\sqrt{29}}{2}$.
\end{crmq}

\subsection{Cercle défini par un diamètre}

\begin{crappel}
Soient $A$ et $B$ deux points. L'ensemble des points $M$ du plan vérifiant $\overrightarrow{MA} \cdot \overrightarrow{MB}=0 $ est le cercle de diamètre $[AB]$.

Autrement dit, le cercle de diamètre $[AB]$ est l'ensemble des points $M$ tels que le triangle $AMB$ est rectangle en $M$.
\end{crappel}

\begin{cdemo}
Notons $\mathscr{C}$ le cercle de diamètre $[AB]$, $I$ le milieu de $[AB]$ et $r$ la distance $AI=IB$.
\begin{flalign*}
	\overrightarrow{MA} \cdot \overrightarrow{MB}=0 & \ssi (\overrightarrow{MI} +\overrightarrow{IA}) \cdot (\overrightarrow{MI} +\overrightarrow{IB}) =0 & \\
	& \ssi \overrightarrow{MI} \cdot \overrightarrow{MI} + \overrightarrow{MI} \cdot \overrightarrow{IB} + \overrightarrow{IA} \cdot \overrightarrow{MI} + \overrightarrow{IA} \cdot \overrightarrow{IB} =0 \\
	& \ssi MI^2 + \overrightarrow{MI} \cdot (\overrightarrow{IA}+\overrightarrow{IB}) - IA \times IB=0\\
	& \ssi MI^2 + \overrightarrow{MI} \cdot \vec{0} - r \times r=0\\
	& \ssi MI^2=r^2\\
	& \ssi MI=r\\
	& \ssi M \in \mathscr{C}
\end{flalign*}
\end{cdemo}

\begin{cexemple}[ v1]
Déterminer une équation cartésienne du cercle de diamètre $[AB]$ où $A(1;-5)$ et $B(3;0)$.

Le milieu de $[AB]$ a pour coordonnées $I\left( \frac{1+3}{2}:\frac{-5+0}{2}\right)=(2;-2,5)$ et $AB=\sqrt{(3-1)^2+(0-(-5))^2}=\sqrt{29}$.

Ainsi $r=\frac{\sqrt{29}}{2}$ et donc $\mathscr{C}$ a donc pour équation cartésienne $(x-2)^2+(y+2,5)^2=\frac{29}{4}$.
\end{cexemple}

\begin{cexemple}[ v2]
Déterminer une équation cartésienne du cercle de diamètre $[AB]$ où $A(1;-5)$ et $B(3;0)$.	
\begin{flalign*}
	M(x;y) \in \mathscr{C} &\ssi \vect{MA} \cdot \vect{MB} = 0 \\ &\ssi \begin{pmatrix}	1-x\\-5-y \end{pmatrix} \cdot \begin{pmatrix} 3-x\\-y \end{pmatrix}=0 & \\&\ssi (1-x)(3-x)+(-5-y)(-y)=0 \\&\ssi 3-x-3x+x^2+5y+y^2=0 \\&\ssi x^2+y^2-4x+5y+3=0
\end{flalign*}
On retrouve bien la même équation de cercle, sous forme développée ou sous forme canonique suivant la méthode utilisée.
\end{cexemple}

\begin{cexercice}
Donner la nature (droite, cercle, parabole, \ldots) des ensembles de points ci-dessous et en préciser les éléments caractéristiques (pour une droite : un point et un vecteur normal ou un vecteur directeur, pour un cercle : le centre et le rayon, \ldots) 

\begin{enumerate}
	\begin{minipage}{0.5\linewidth}
		\item $5x+4y-3=0$
		\item $x=5$
		\item $(x-3)^2+(y+7)^2=12$
		\item $x^2+y^2=1$
		\item $(x-2)+(y-5)=4$
	\end{minipage}  \hfill 	\begin{minipage}{0.5\linewidth}
		\item $3x-5=2y+1$
		\item $x^2+y^2+6x+2y-6=0$
		\item $y=3x^2-5x+1$
		\item $x^2-y^2=0$
	\end{minipage}
\end{enumerate}
\end{cexercice}

\end{document}