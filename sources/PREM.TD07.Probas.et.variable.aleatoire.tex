% !TeX TXS-program:compile = txs:///arara
% arara: lualatex: {shell: no, synctex: yes, interaction: batchmode}
% arara: pythontex: {rerun: modified} if found('pytxcode', 'PYTHONTEX#py')
% arara: lualatex: {shell: no, synctex: yes, interaction: batchmode} if found('pytxcode', 'PYTHONTEX#py')
% arara: lualatex: {shell: no, synctex: yes, interaction: batchmode} if found('log', '(undefined references|Please rerun|Rerun to get)')

\documentclass[a4paper,11pt]{article}
\usepackage[revgoku]{cp-base}
\graphicspath{{./graphics/}}
%variables
\donnees[%
	classe=1\up{ère} 2M2,
	matiere={[SPÉ.MATHS]},
	typedoc=TD,
	numdoc=07,
	mois=Mai,
	annee=2022
	]

%formatage
\author{Pierquet}
\title{\nomfichier}
\hypersetup{pdfauthor={Pierquet},pdftitle={\nomfichier},allbordercolors=white,pdfborder=0 0 0,pdfstartview=FitH}
%divers
\lhead{\entete{\matiere}}
\chead{\entete{\lycee}}
\rhead{\entete{\classe{} - \mois{} \annee}}
%\rhead{\entete{\classe{} - Chapitre }}
\lfoot{\pied{\matiere}}
\cfoot{\logolycee{}}
\rfoot{\pied{\numeropagetot}}

\begin{document}

\pagestyle{fancy}

\part{TD07 - Probabilités et variable aléatoire}

\smallskip

%\nomprenomtcbox
%
%\medskip

On dispose d'un paquet de cartes contenant un nombre identique de cartes de la catégorie \og Sciences \fg{} et de la catégorie \og Économie \fg. Une question liée à un de ces deux thèmes figure sur chaque carte.

Les cartes sont mélangées et on en tire une au hasard dans le paquet. Ensuite, on essaye de répondre à la question posée.

\smallskip

Un groupe de copains participe à ce jeu. Connaissant leurs points forts et leurs faiblesses, on estime qu'il a :

\begin{itemize}
	\item 3 chances sur 4 de donner la bonne réponse lorsqu'il est interrogé en sciences ;
	\item 1 chance sur 8 de donner la bonne réponse lorsqu'il est interrogé en économie.
\end{itemize}

On note $S$ l'évènement \og La question est dans la catégorie Sciences \fg{} et $B$ l'évènement \og La réponse donnée par le groupe est bonne \fg.

\bigskip

\textbf{-- Partie A --}
%
\begin{enumerate}
	\item Construire un arbre pondéré modélisant la situation présentée dans l'énoncé.
	\item Calculer $P(B \cap S)$.
	\item Déterminer la probabilité que le groupe de copains réponde correctement à la question posée.
	\item Calculer $P_B(E)$ et interpréter le résultat dans le contexte de l'exercice.
	\item Les évènements $S$ et $B$ sont-ils indépendants ? Justifier.
	\item On choisit successivement, et avec remise, deux cartes du paquet. Déterminer la probabilité que la bonne réponse est donnée pour les deux cartes.
\end{enumerate}

\textbf{-- Partie B --}

\medskip

Pour participer à ce jeu, on doit payer 5\,€ de droit d'inscription. On recevra :

\begin{itemize}
	\item 10\,€ si on est interrogé en sciences et que la réponse est correcte ;
	\item 30\,€ si on est interrogé en économie et que la réponse est correcte ;
	\item rien si la réponse donnée est fausse.
\end{itemize}

Soit $X$ la variable aléatoire qui, à chaque partie jouée, associe son gain. On appelle gain la différence en euros entre ce qui est reçu et les 5\,€ de droit d'inscription.

\begin{enumerate}
	\item Déterminer (sous forme de tableau) la loi de probabilité de $X$.
	\item On considère la fonction \cpy{jeu} ci-dessous en langage \calgpython, pour laquelle les paramètres \cpy{L} et \cpy{G} sont des listes.
		
	\begin{envpython}[13cm]
		def jeu(L,G) :
			n = len(L)
			E = 0
			for i in range(n) :
				E = E + L[i]*G[i]
			return E
	\end{envpython}
	\begin{enumerate}
		\item Que retourne la fonction \cpy{jeu} avec comme paramètres les listes ci-dessous ?
		\begin{itemize}
			\item \cpy{L = [-5 , 5 , 25]} ;
			\item \cpy{G = [0.5625 , 0.375 , 0.0625]}
		\end{itemize}
		\item Interpréter le résultat dans le contexte de l'exercice.
	\end{enumerate}
\end{enumerate}






\end{document}