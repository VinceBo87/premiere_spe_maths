% !TeX TXS-program:compile = txs:///arara
% arara: lualatex: {shell: no, synctex: yes, interaction: batchmode}
% arara: pythontex: {rerun: modified} if found('pytxcode', 'PYTHONTEX#py')
% arara: lualatex: {shell: no, synctex: yes, interaction: batchmode} if found('pytxcode', 'PYTHONTEX#py')
% arara: lualatex: {shell: no, synctex: yes, interaction: batchmode} if found('log', '(undefined references|Please rerun|Rerun to get)')

\documentclass[a4paper,11pt]{article}
\usepackage[revgoku]{cp-base}
\graphicspath{{./graphics/}}
%variables.
\donnees[classe={1\up{ère} 2M2},matiere={[SPÉ.MATHS]},mois={Jeudi 24 Mars},annee=2022,typedoc=TEST~,numdoc=7,duree={15 minutes}]
%formatage
\author{Pierquet}
\title{\nomfichier}
\hypersetup{pdfauthor={Pierquet},pdftitle={\nomfichier},allbordercolors=white,pdfborder=0 0 0,pdfstartview=FitH}
%divers
\lhead{\entete{Durée : \duree}}
\chead{\entete{\lycee}}
\rhead{\entete{\classe{} - \mois{} \annee}}
\lfoot{\pied{\matiere}}
\cfoot{\logolycee{}}
\fancypagestyle{sujetA}{\fancyhead[R]{\entete{\classe{}A - \mois{} \annee}}}
\fancypagestyle{sujetB}{\fancyhead[R]{\entete{\classe{}B - \mois{} \annee}}}

\begin{document}

\pagestyle{fancy}


\part{TEST07 - Suites, v2}%SUJETA

\setcounter{numexos}{0}

\medskip

\nomprenomtcbox

\medskip

\exonum{}

\medskip

On considère la suite $\suiten$ définie par $\begin{dcases} u_0 = 10 \\ u_{n+1} = u_n + 3 \text{ pour tout entier }n \end{dcases}$.

\begin{enumerate}
	\item Déterminer, en justifiant, la nature de la suite $\suiten$.
	\item Déterminer la formule explicite de $u_n$ en fonction de $n$.
	\item Calculer $u_{10}$.
	\item Déterminer, en justifiant, le sens de variation de la suite $\suiten$.
\end{enumerate}

\papierseyes{21}{9}

\medskip

\exonum{}

\medskip

On considère la suite $\suiten[v]$ géométrique de premier terme $v_1=100$ et de raison $q=0,95$.

\begin{enumerate}
	\item Donner la formule de récurrence liée à la suite $\suiten[v]$.
	\item Déterminer la formule explicite donnant $v_n$ en fonction de $n$.
	\item En déduire, arrondie au millième, la valeur de $v_9$.
\end{enumerate}

\papierseyes{21}{8}

\end{document}