% !TeX TXS-program:compile = txs:///pythonlualatex

\documentclass[a4paper,11pt]{article}
\usepackage[pythontex,revgoku]{cp-base}
\graphicspath{{./graphics/}}
%variables
\donnees[classe=1\up{ère} 2M2,matiere={[SPÉ.MATHS]},mois=Janvier,annee=2022,typedoc=DM,numdoc=5,titre={Probabilités conditionnelles}]

%formatage
\author{Pierquet}
\title{\nomfichier}
\hypersetup{pdfauthor={Pierquet},pdftitle={\nomfichier},allbordercolors=white,pdfborder=0 0 0,pdfstartview=FitH}
%divers
\lhead{\entete{\matiere}}
\chead{\entete{\lycee}}
\rhead{\entete{\classe{} - \mois{} \annee}}
\lfoot{\pied{\matiere}}
\cfoot{\logolycee{}}
\rfoot{\pied{\numeropagetot}}
\fancypagestyle{entetedm}{\fancyhead[L]{\entete{\matiere{} À rendre avant le\ldots}}}

\begin{document}

\pagestyle{fancy}

\thispagestyle{entetedm}

\setcounter{numexos}{0}

\part{DM05 - Probabilités conditionnelles}%barbazo exo37p296

\smallskip

\exonum{2}

\medskip

Une entreprise conditionne du sucre blanc provenant de deux exploitations U et V en paquets de 1 kg et de différentes qualités. Le sucre extra fin est conditionné séparément dans des paquets portant le label « extra fin ».

\smallskip

Les résultats seront arrondis, si nécessaire, au millième.

\smallskip

On sait que 3\,\% du sucre provenant de l’exploitation U est extra fin et 5\,\% du sucre provenant de l’exploitation V est extra fin. On prélève au hasard un paquet de sucre dans la production de l’entreprise et, dans un souci de traçabilité, on s’intéresse à la provenance de ce paquet.

\smallskip

On considère les évènements suivants :
%
\begin{itemize}
	\item U : « Le paquet contient du sucre provenant de l’exploitation U » ;
	\item V : « Le paquet contient du sucre provenant de l’exploitation V » ;
	\item E : « Le paquet porte le label "extra fin" ».
\end{itemize}
%
\begin{enumerate}
	\item Dans cette question, on admet que l’entreprise fabrique 30\,\% de ses paquets avec du sucre provenant de l’exploitation U et les autres avec du sucre provenant de l’exploitation V, sans mélanger les sucres des deux exploitations.
	\begin{enumerate}
		\item Construire un arbre pondéré représentant la situation.
		\item Quelle est la probabilité que le paquet prélevé porte le label « extra fin » ? 
		\item Sachant qu’un paquet porte le label « extra fin », quelle est la probabilité que le sucre qu’il contient provienne de l’exploitation U ? 
	\end{enumerate}
	\item L’entreprise souhaite modifier son approvisionnement auprès des deux exploitations afin que, parmi les paquets portant le label « extra fin », 30\,\% d’entre eux contiennent du sucre provenant de l’exploitation U. On cherche à déterminer comment elle doit s’approvisionner auprès des 
	exploitations U et V.
	
	On ne connaît donc pas, dans cette question, $P(U)$, et on va noter $P(U)=x$.
	\begin{enumerate}
		\item Construire, à l'aide de $x$, un arbre de pondéré représentant la situation.
		\item Déterminer, en détaillant la démarche, les valeurs de $P(U)$ et de $P(V)$.
	\end{enumerate}
\end{enumerate}

\medskip

\exonum{2}

\medskip

Une urne contient \textcolor{ForestGreen}{cinq bulletins verts} et {\blue trois bulletins bleus}. Lucien a écrit le script \calgpython{} d’une fonction \cpy{tirage}.

\begin{envpython}[12cm]
	from random import randint
	def tirage():
		tirage=[]
		for i in range(2):
			if randint(1,8)<=3:
				tirage.append("bleu")
			else :
				tirage.append("vert")
		return(tirage)
\end{envpython}

\begin{enumerate}
	\item Que renvoie cette fonction ?
	\item Quelle est la probabilité que cette fonction renvoie \cpy{['bleu','bleu']} ?
	\item[Bonus] Modifier le script de la fonction \cpy{tirage} pour qu’elle renvoie le nombre de bulletins bleus obtenus lors de deux tirages avec remise dans l’urne.
\end{enumerate}

\end{document}