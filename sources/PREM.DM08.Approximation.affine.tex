% !TeX TXS-program:compile = txs:///lualatex

\documentclass[a4paper,11pt]{article}
\usepackage[revgoku,breakable]{cp-base}
\graphicspath{{./graphics/}}
%variables
\donnees[classe=1\up{ère} 2M2,matiere={[SPÉ.MATHS]},mois=Mars,annee=2022,typedoc=DM,numdoc=8,titre={Approximation affine}]

%formatage
\author{Pierquet}
\title{\nomfichier}
\hypersetup{pdfauthor={Pierquet},pdftitle={\nomfichier},allbordercolors=white,pdfborder=0 0 0,pdfstartview=FitH}
%divers
\lhead{\entete{\matiere}}
\chead{\entete{\lycee}}
\rhead{\entete{\classe{} - \mois{} \annee}}
%\rhead{\entete{\classe{} - Chapitre }}
\lfoot{\pied{\matiere}}
\cfoot{\logolycee{}}
\rfoot{\pied{\numeropagetot}}
\fancypagestyle{entetedm}{\fancyhead[L]{\entete{\matiere{} À rendre avant le\ldots}}}

\begin{document}

\pagestyle{fancy}

\thispagestyle{entetedm}

\part{DM08 - Approximation affine}

\smallskip

\begin{cidee}
Dans ce DM, on va travailler sur l'\textbf{approximation affine} (notée \textbf{AP}) d'une fonction au \textit{voisinage} d'un point.

On considère une fonction $f$ définie sur un intervalle $I$ contenant un réel $a$. Les résultats vus au chapitre 07 ont permis de mettre en évidence le fait que si $f$ était dérivable en $a$, alors la courbe $\mathscr{C}_f$ est \textit{très proche} de la tangente $\mathscr{T}_a$ autour du point A de $\mathscr{C}_f$ d'abscisse $a$.

\tunits{0.9}{0.9}
\tdefgrille{0}{5}{1}{0.5}{0}{3}{1}{0.5}
\begin{tikzpicture}[x=\xunit cm,y=\yunit cm]
	\axestikz*[width=1pt] ;
	\foreach \x in {0,1} \draw[line width=1pt] (\x,4pt) -- (\x,-4pt);
	\foreach \y in {0,1} \draw[line width=1pt] (4pt,\y) -- (-4pt,\y);
	\draw (3,0) node[below,darkgray] {$\mathsf{a}\vphantom{\mathsf{h}}$} ;
	\draw (4.25,0) node[below,ForestGreen] {$\mathsf{a+h}$} ;
	\draw[line width=1pt,densely dashed,darkgray] (3,0) |- (0,1.25) node[left,darkgray] {$\mathsf{f(a)}$} ;
	\draw[line width=1pt,densely dashed,ForestGreen] (4.25,0) |- (0,2.265625) node[left,ForestGreen] {$\mathsf{f(a+h)}$} ;
	\draw[line width=1pt,densely dashed,purple] (4.25,1.875) -- (0,1.875) node[left,purple] {$\mathsf{y_N}$} ;
	\draw (4.85,2.85) node[red,right] {\large $\mathscr{C}_f$} ;
	\draw (5,2.25) node[blue,right] {\large $\mathscr{T}_a$} ;
	%COURS
	\draw (5.75,2.5) node[right=0pt] {On sait que $\lim_{h \to 0} \dfrac{f(a+h)-f(a)}{h}=f'(a)$.} ;
	\draw (5.75,1.25) node[right=0pt] {Donc pour $h$ proche de $0$, $\textcolor{ForestGreen}{\underbrace{f(a+h)}_{y_M}}$ est proche de $\textcolor{purple}{\underbrace{f'(a) \times h + f(a)}_{y_N}}$.} ;
	\draw (5.75,0.25) node[right=0pt] {Autrement dit, on a {\red $f(a+h) \approx f'(a) \times h + f(a)$}.} ;
	%COURBES
	\clip (\xmin,\ymin) rectangle (\xmax,\ymax) ;
	\draw[line width=1.25pt,red,domain=\xmin:\xmax,samples=250] plot (\x,{0.25*(\x-2)^2+1}) ;
	\draw[line width=1.25pt,blue,samples=2,domain=\xmin:\xmax] plot (\x,{0.5*(\x-3)+1.25}) ;
	\filldraw[darkgray] (3,1.25) circle[radius=2pt] node[above left,darkgray] {\sf A} ;
	\filldraw[ForestGreen] (4.25,2.265625) circle[radius=2pt] node[above left,ForestGreen] {\sf M} ;
	\filldraw[purple] (4.25,1.875) circle[radius=2pt] node[below right,purple] {\sf N} ;
\end{tikzpicture}

%\smallskip

\hfill~On obtient donc une \textbf{approximation affine} de $f$ au voisinage de $a$ : {\red\fbox{$\grasmaths{f(a+h) \approx f'(a) \times h + f(a)}$}}.\hfill~
\end{cidee}

\begin{cexemple}
Par exemple, si on considère la fonction $f(x)=x^2$ au voisinage de $a=3$ :
%
\begin{itemize}
	\item on a $f(3)=3^2=9$ ;
	\item de plus on sait que $f'(x)=2x$ de sorte de $f'(3)=2 \times 3 = 6$ ;
	\item une approximation affine de $f$ au voisinage de $a=3$ est donc $f(3+h) \approx f'(3) \times h + f(3) \approx {\red 6h + 9}$ ;
	\item en \og revenant \fg{} à l'expression de $f$, on obtient donc finalement ${\red (3+h)^2 \approx 6h + 9}$ pour $h$ très petit.
\end{itemize}
%
Ainsi, on peut exploiter le résultat précédent :
%
\begin{itemize}
	\item pour $h=0,01$ on obtient $f(3,01) \approx 6 \times 0,01 + 9$ soit $3,01^2 \approx 9,06$ en sachant que $3,01^2 = 9,0601$ !
	\item pour $h=0,001$ on obtient $f(3,001) \approx 6 \times 0,001 + 9$ soit $3,001^2 \approx 9,006$ en sachant que $3,001^2 = 9,006001$ !
\end{itemize}
\end{cexemple}

\begin{cblocdm}[ n°1]
On s'intéresse à la fonction $f$ définie et dérivable sur $\R$ par $f(x)=x^3$, au voisinage de $a=2$.
%
\begin{enumerate}[noitemsep,topsep=4pt]
	\item Rappeler la formule donnant $f'(x)$.
	\item Démontrer qu'une approximation affine de $f$ au voisinage de $a=2$ est donnée par $(2+h)^3 \approx 12h+8$.
	\item En déduire, \uline{en utilisant l'AP précédente}, une valeur approchée de $2,1^3$ puis de $1,99^3$.
\end{enumerate}
\end{cblocdm}

\begin{cblocdm}[ n°2]
On s'intéresse à la fonction $g$ définie sur $\intervFO{0}{+\infty}$ et dérivable sur $\intervOO{0}{+\infty}$ par $g(x)=\sqrt{x}$, au voisinage de $a=4$.
%
\begin{enumerate}[noitemsep,topsep=4pt]
	\item Rappeler la formule donnant $g'(x)$.
	\item En utilisant l'\textbf{AP} de $g$ au voisinage de $a=4$, déterminer une valeur approchée de $\sqrt{4+h}$ pour $h$ petit.
	\item En déduire, \uline{en utilisant l'approximation affine}, une valeur approchée de $\sqrt{4,1}$ puis de $\sqrt{3,999}$.
\end{enumerate}
\end{cblocdm}

\begin{cblocdm}[ n°3]
On s'intéresse à la fonction $k$ définie et dérivable sur $\R^*$ par $k(x)=\dfrac{1}{x^2}$, au voisinage de $a=1$.
%
\begin{enumerate}[noitemsep,topsep=4pt]
	\item Rappeler la formule donnant $k'(x)$.
	\item En utilisant l'\textbf{AP} de $k$ au voisinage de $a=1$, donner une valeur approchée de $\dfrac{1}{(1+h)^2}$ pour $h$ très petit.
	\item En déduire, \uline{en utilisant l'approximation affine précédente}, une valeur approchée de $\dfrac{1}{1,1^2}$ puis de $\dfrac{1}{0,99^2}$.
\end{enumerate}
\end{cblocdm}

\end{document}