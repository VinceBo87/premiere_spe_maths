% !TeX TXS-program:compile = txs:///lualatex

\documentclass[a4paper,11pt]{article}
\usepackage[revgoku]{cp-base}
\graphicspath{{./graphics/}}
%variables.
\donnees[classe={1\up{ère} 2M2},matiere={[SPÉ.MATHS]},mois={Mardi 12 Octobre},annee=2021,typedoc=TEST~,numdoc=2,duree={15 minutes}]
%formatage
\author{Pierquet}
\title{\nomfichier}
\hypersetup{pdfauthor={Pierquet},pdftitle={\nomfichier},allbordercolors=white,pdfstartview=FitH}
%divers
\lhead{\entete{Durée : \duree}}
\chead{\entete{\lycee}}
\rhead{\entete{\classe{} - \mois{} \annee}}
\lfoot{\pied{\matiere}}
\cfoot{\logolycee{}}
%\rfoot{\pied{\numeropagetot}}
\fancypagestyle{sujetA}{\fancyhead[R]{\entete{\classe{}A - \mois{} \annee}}}
\fancypagestyle{sujetB}{\fancyhead[R]{\entete{\classe{}B - \mois{} \annee}}}

\begin{document}

\pagestyle{fancy}

\part{TEST02 - Suites, v1}

\medskip

\nomprenomtcbox

\begin{enumerate}
	\item On considère la suite $\suiten$ définie par $u_n = \dfrac{2}{n}+1$ pour tout entier naturel $n$ non nul.
	\begin{enumerate}
		\item La suite $\suiten$ est-elle définie de manière explicite ou par récurrence ? Justifier.
		
		\papierseyes{20}{2}
		
		\item Calculer les valeurs de $u_1$, $u_2$ et $u_3$.
		
		\papierseyes{20}{3}
		
		\item Déterminer une expression de la fonction $f$ telle que $u_n=f(n)$.
		
		\papierseyes{20}{2}
		
		\item Déterminer le 10\ieme{} terme de la suite $\suiten$.
		
		\papierseyes{20}{2}
	\end{enumerate}
	\item On considère la suite $\suiten[v]$ définie par $\begin{dcases} v_0 = 4 \\ v_{n+1}=3v_n-5 \end{dcases}$ pour tout entier naturel $n$.
	\begin{enumerate}
		\item La suite $\suiten[v]$ est-elle définie de manière explicite ou par récurrence ? Justifier.
		
		\papierseyes{20}{2}
		
		\item Calculer les valeurs de $v_1$, $v_2$ et $v_3$.
		
		\papierseyes{20}{4}
		
		\item En utilisant la calculatrice, déterminer la valeur de $v_{7}$.
		
		\papierseyes{20}{2}
		
	\end{enumerate}
\end{enumerate}

\end{document}