% !TeX TXS-program:compile = txs:///lualatex

\documentclass[a4paper,11pt]{article}
\usepackage[revgoku]{cp-base}
\graphicspath{{./graphics/}}
%variables
\donnees[%
	classe=1\up{ère} 2M2,matiere={[SPÉ.MATHS]},typedoc=TD~,numdoc=01,mois=Septembre,titre={Second degré}
	]

%formatage
\author{Pierquet}
\title{\nomfichier}
\hypersetup{pdfauthor={Pierquet},pdftitle={\nomfichier},allbordercolors=white,pdfborder=0 0 0,pdfstartview=FitH}
\lhead{\entete{\matiere}}
\chead{\entete{\lycee}}
\rhead{\entete{\classe{} - \mois{} \annee}}
\lfoot{\pied{\matiere}}
\cfoot{\logolycee{}}
\rfoot{\pied{\numeropagetot}}
%divers

\begin{document}

\pagestyle{fancy}

%\vspace*{-0.8\baselineskip}

\part{Second degré (Correction)}

\medskip

\exonum{0}

\begin{enumerate}
	\item On utilise la formule $\alpha=-\dfrac{b}{2a}$ ; $\beta=a\alpha^2+b\alpha+c$ et $ax^2+bx+c=a(x-\alpha)^2+\beta$ :
	\begin{enumerate}
		\item $f(x)=-2x^2+4x+7 = -2(x-1)^2+9$ avec sommet en $(1\,;\,9)$ et $\nearrow\searrow$.
		\item $g(x)=-6x^2+9x+13=-6\left(x-\dfrac34\right)^2+\dfrac{131}{8}$ avec sommet en $\left(\dfrac34\,;\,\dfrac{131}{8}\right)$ et $\nearrow\searrow$.
		\item $h(x)=0,5x^2-10x=0,5(x-10)^2-50$ avec sommet en $(10\,;\,-50)$ et $\searrow\nearrow$.
	\end{enumerate}
	\item On utilise $\Delta=b^2-4ac$ :
	\begin{enumerate}
		\item $f(x)=7x^2-7x+8$ : $\Delta=-175$ donc par de racine et pas de factorisation.
		\item $g(x)=-x^2+10x-25$ : $\Delta=0$ et $x_0=5$ et $g(x)=-(x-5)^2$.
		\item $h(x)=-2x^2+x+1$ : $\Delta=9$ et $x_1 = -\dfrac12=-0,5$ et $x_2=1$ et $h(x)=-2(x+0,5)(x-1)$.
	\end{enumerate}
	\item On résout (après avoir obtenu \og $=0$ \fg) grâce à $\Delta$ :
	\begin{enumerate}
		\item $2x^2-12x+10=0$ : $\Delta=64$ et $x_1=1$ et $x_2=5$. Donc $\mathscr{S}=\left\lbrace \strut 1\,;\,5 \right\rbrace$.
		\item $-2x^2+4x-6=0$ : $\Delta=-32$. Donc $\mathscr{S}=\varnothing$.
		\item $0,25x^2+3,5x+12,25=0$ : $\Delta=0$ et $x_0=-7$. Donc $\mathscr{S}=\left\lbrace \strut -7 \right\rbrace$.
	\end{enumerate}
\end{enumerate}

\medskip

\exonum{1}

\medskip

On résout les équations en les transformant si besoin :
%
\begin{enumerate}
	\item $(x-3)(x^2+6x+5)=0$ est une équation produit (nul) :
	\begin{itemize}
		\item $x-3=0 \ssi x=3$ ;
		\item $x^2+6x+5=0$ : $\Delta=1$ et $x_1=-1$ et $x_2=-5$.
	\end{itemize}
	Donc $\mathscr{S}=\left\lbrace \strut 3\,;\,-1\,;\,-5 \right\rbrace$.
	\item $3x^2+4x-5 = x^2 +4x+3 \ssi 2x^2-8 = 0$ : $\Delta=64$ et $x_1=2$ et $x_2=-2$. Donc $\mathscr{S}=\left\lbrace \strut 2\,;\,-2 \right\rbrace$.
	\item $\dfrac{3}{x+5}=x-3$ (pour $x \neq -5$) est une équation quotient, donc produit en croix :
	
	$\dfrac{3}{x+5}=x-3 \rightarrow 3 \times 1 = (x+5)(x-3) \Rightarrow 3 = x^2-3x+5x-15 \Rightarrow 0 = x^2+2x-18=0$.
	
	$\Delta=76$ et $x_1=-1+\sqrt{19}$ et $x_1=-1-\sqrt{19}$ (pas valeur interdite). Donc $\mathscr{S}=\left\lbrace \strut -1+\sqrt{19}\,;\,-1-\sqrt{19} \right\rbrace$.
\end{enumerate}

\medskip

\exonum{2}

\begin{center}
	\tunits{0.55}{0.55}
	\tdefgrille{-1}{7}{1}{0.5}{-3}{7}{1}{0.5}
	\begin{tikzpicture}[x=\xunit cm,y=\yunit cm]
		\tgrillep[densely dashed,line width=0.6pt,gray!50]
		\draw[->,line width=1.25pt] (\xmin,0) -- (\xmax,0);
		\draw[->,line width=1.25pt] (0,\ymin) -- (0,\ymax);
		\foreach \x in {-1,0,...,6} %à compléter avec itération ou complètement
			\draw[line width=1.25pt] (\x,4pt) -- (\x,-4pt) node[below] {\scriptsize \num{\x}}; %éventuellement \xx et taille...
		\foreach \y in {-3,-2,...,6}
			\draw[line width=1.25pt] (4pt,\y) -- (-4pt,\y) node[left] {\scriptsize \num{\y}};
		\draw[line width=1.25pt,red,domain=-1:7,samples=200] plot(\x,{0.5*(\x-3)*(\x-3)-2});
		\foreach \Point/\Color in {(1,0)/blue,(5,0)/blue,(3,-2)/ForestGreen}
		\draw[\Color,fill=\Color] \Point circle[radius=3pt];
	\end{tikzpicture}
\end{center}

\begin{enumerate}
	\item
	\begin{enumerate}
		\item Graphiquement, on obtient (grâce au sommet $S(3\,;\,-2)$) que $\alpha=3$ et $\beta=-2$. \hfill\cgrph{\noticetikz*{ForestGreen}}
		
		Les points d'intersection de la parabole avec l'axe des abscisses donne les racines $x_1=1$ et $x_2=5$.  \hfill\cgrph{\noticetikz*{blue}}
		\item Le coefficient $a$ est strictement positif, car la parabole est \og ouverte vers le haut \fg.
		\item De plus, on peut affirmer que $\Delta>0$ car la fonction $f$ admet deux racines.
	\end{enumerate}
	\item 
	\begin{enumerate}
		\item Forme canonique : $f(x)=a(x-3)^2-2$ et $f(1)=0 \ssi a(1-3)^2-2=0 \ssi a \times 4 = 2 \ssi a=0,5$.
		\item Forme factorisée : $f(x)=a(x-1)(x-5)$ et $f(3)=-2 \ssi a(3-1)(3-5)=-2 \ssi a \times (-4)=-2 \ssi a=0,5$.
	\end{enumerate}
\end{enumerate}

\medskip

\exonum{1}

\begin{multicols}{2}
	\begin{enumerate}
		\item $\Delta = 0$ et $a > 0$ : \og {\red souriante} \fg{} et {\blue 1 racine}
		\begin{center}
			\begin{tikzpicture}[x=0.8cm,y=0.8cm,xmin=-3,xmax=3,ymin=-3,ymax=3]
				\tgrillep[densely dashed,line width=0.6pt,gray!50] \axestikz*
				\axextikz*{-3,-2,...,2} \axeytikz*{-3,-2,...,2}
				\clip (\xmin,\ymin) rectangle (\xmax,\ymax) ;
				\draw[line width=1.25pt,red,samples=250,domain=\xmin:\xmax] plot (\x,{1.5*(\x-1.5)^2}) ;
				\filldraw[blue] (1.5,0) circle[radius=3pt] ;
			\end{tikzpicture}
		\end{center}
		\item $\Delta > 0$ et $a > 0$ : \og {\red souriante} \fg{} et {\blue 2 racines}
		\begin{center}
			\begin{tikzpicture}[x=0.8cm,y=0.8cm,xmin=-3,xmax=3,ymin=-3,ymax=3]
				\tgrillep[densely dashed,line width=0.6pt,gray!50] \axestikz*
				\axextikz*{-3,-2,...,2} \axeytikz*{-3,-2,...,2}
				\clip (\xmin,\ymin) rectangle (\xmax,\ymax) ;
				\draw[line width=1.25pt,red,samples=250,domain=\xmin:\xmax] plot (\x,{0.85*(\x-1)*(\x+2)}) ;
				\filldraw[blue] (1,0) circle[radius=3pt] (-2,0) circle[radius=3pt];
			\end{tikzpicture}
		\end{center}
		\item $\Delta < 0$ et $a > 0$ : \og {\red souriante} \fg{} et {\blue 0 racine}
		\begin{center}
			\begin{tikzpicture}[x=0.8cm,y=0.8cm,xmin=-3,xmax=3,ymin=-3,ymax=3]
				\tgrillep[densely dashed,line width=0.6pt,gray!50] \axestikz*
				\axextikz*{-3,-2,...,2} \axeytikz*{-3,-2,...,2}
				\clip (\xmin,\ymin) rectangle (\xmax,\ymax) ;
				\draw[line width=1.25pt,red,samples=250,domain=\xmin:\xmax] plot (\x,{\x*\x+\x+1}) ;
			\end{tikzpicture}
		\end{center}
		\item $\Delta = 0$ et $a < 0$ : \og {\red non souriante} \fg{} et {\blue 1 racine}
		\begin{center}
			\begin{tikzpicture}[x=0.8cm,y=0.8cm,xmin=-3,xmax=3,ymin=-3,ymax=3]
				\tgrillep[densely dashed,line width=0.6pt,gray!50] \axestikz*
				\axextikz*{-3,-2,...,2} \axeytikz*{-3,-2,...,2}
				\clip (\xmin,\ymin) rectangle (\xmax,\ymax) ;
				\draw[line width=1.25pt,red,samples=250,domain=\xmin:\xmax] plot (\x,{-2*(\x+1.75)^2}) ;
				\filldraw[blue] (-1.75,0) circle[radius=3pt] ;
			\end{tikzpicture}
		\end{center}
		\item $\Delta > 0$ et $a < 0$ : \og {\red non souriante} \fg{} et {\blue 2 racines}
		\begin{center}
			\begin{tikzpicture}[x=0.8cm,y=0.8cm,xmin=-3,xmax=3,ymin=-3,ymax=3]
				\tgrillep[densely dashed,line width=0.6pt,gray!50] \axestikz*
				\axextikz*{-3,-2,...,2} \axeytikz*{-3,-2,...,2}
				\clip (\xmin,\ymin) rectangle (\xmax,\ymax) ;
				\draw[line width=1.25pt,red,samples=250,domain=\xmin:\xmax] plot (\x,{-1.5*(\x+2)*(\x+0.5)}) ;
				\filldraw[blue] (-2,0) circle[radius=3pt] (-0.5,0) circle[radius=3pt];
			\end{tikzpicture}
		\end{center}
		\item $\Delta < 0$ et $a < 0$ : \og {\red non souriante} \fg{} et {\blue 0 racine}
		\begin{center}
			\begin{tikzpicture}[x=0.8cm,y=0.8cm,xmin=-3,xmax=3,ymin=-3,ymax=3]
				\tgrillep[densely dashed,line width=0.6pt,gray!50] \axestikz*
				\axextikz*{-3,-2,...,2} \axeytikz*{-3,-2,...,2}
				\clip (\xmin,\ymin) rectangle (\xmax,\ymax) ;
				\draw[line width=1.25pt,red,samples=250,domain=\xmin:\xmax] plot (\x,{-0.5*\x*\x-2*\x-2.25}) ;
			\end{tikzpicture}
		\end{center}
	\end{enumerate}
\end{multicols}


\end{document}