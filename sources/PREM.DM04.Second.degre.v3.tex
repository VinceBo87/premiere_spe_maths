% !TeX TXS-program:compile = txs:///pythonlua

\documentclass[a4paper,11pt]{article}
\usepackage[pythontex,revgoku]{cp-base}
\graphicspath{{./graphics/}}
%variables
\donnees[%
	classe=1\up{ère} 2M2,
	matiere={[SPÉ.MATHS]},
	mois=Décembre,
	annee=2021,
	typedoc=DM,
	numdoc=04,
	titre={Second degré en situation(s)}
	]

%formatage
\author{Pierquet}
\title{\nomfichier}
\hypersetup{pdfauthor={Pierquet},pdftitle={\nomfichier},allbordercolors=white,pdfborder=0 0 0,pdfstartview=FitH}
%divers
\lhead{\entete{\matiere}}
\chead{\entete{\lycee}}
\rhead{\entete{\classe{} - \mois{} \annee}}
\lfoot{\pied{\matiere}}
\cfoot{\logolycee{}}
\rfoot{\pied{\numeropagetot}}
\fancypagestyle{entetedm}{\fancyhead[L]{\entete{\matiere{} À rendre avant le\ldots}}}

\begin{document}

\pagestyle{fancy}

\thispagestyle{entetedm}

\part{DM04 - Second degré en situation(s)}

\smallskip

\exotitre{Exercice 1 - En SES}

\medskip

Une entreprise fabrique des composants électroniques. Sa production mensuelle est inférieure à 12\,000 articles.

Le coût total mensuel, en milliers d’euros, pour produire $x$ milliers d’articles est modélisé par la fonction C définie sur $\intervFF{0}{12}$ par $C(x) = 0,6x^2 - 0,62x + 18,24$. Chaque article fabriqué est vendu au prix unitaire de 7\,€.

\begin{enumerate}
	\item L’entreprise a produit et vendu 4\,000 articles en mai 2018 et 6\,500 articles en juin 2018.
	
	Le bénéfice a-t-il été plus important au mois de juin ?
	\item On note $R(x)$ le montant, en milliers d’euros, de la recette mensuelle pour $x$ milliers d’articles vendus.
	
	Exprimer $R(x)$ en fonction de $x$.
	\item Tracer soigneusement, dans le repère donné en annexe (les unités sont à préciser), les courbes représentatives $\mathscr{C}$ et $\mathscr{R}$ des fonctions C et R. 
	\item Avec la précision permise par le graphique, déterminer :
	\begin{enumerate}
		\item l’intervalle dans lequel doit se situer $x$ pour que le bénéfice mensuel réalisé soit positif ;
		\item la valeur de $x$ pour laquelle le bénéfice mensuel est maximal.
	\end{enumerate}
	\item On note $B(x)$ le bénéfice mensuel, en milliers d’euros, réalisé lorsque l’entreprise produit et vend $x$ milliers d’articles.
	\begin{enumerate}
		\item Vérifier que, pour tout $x \in \intervFF{0}{12}$, on a $B(x) = -0,6x^2 + 7,62x - 18,24$.
		\item Étudier le signe de $B(x)$ et les variations de la fonction B sur $\intervFF{0}{12}$.
	\end{enumerate}
	\item En déduire le nombre d’articles que l’entreprise doit produire pour réaliser un bénéfice mensuel :
	\begin{enumerate}
		\item positif ; 
		\item maximal.
	\end{enumerate}
\end{enumerate}

\medskip

\exotitre{Exercice 2 - En programmation}

\medskip

On considère la fonction suivante en \calgpython{} qui fait intervenir une fonction $f$ polynôme du second degré :

\begin{tcpythoncode}[10cm]
	\begin{pyverbatim}[][fontsize=\footnotesize,numbers=left,numbersep=10pt]
		def point_f(x,y) :
			if y == 3*x**2-x-2 :
				return True
			else :
				return False
	\end{pyverbatim}
\end{tcpythoncode}

\begin{enumerate}
	\item Donner une expression de $f(x)$.
	\item On appelle la fonction \cpy{point\_f} avec les paramètres \cpy{x=-1/3} et \cpy{y=-29/3}. Que renvoie-t-elle ?
	\item On appelle la fonction \cpy{point\_f} avec les paramètres \cpy{x=-1} et \cpy{y=2}. Que renvoie-t-elle ?
	\item Préciser le rôle de cette fonction en \calgpython.
	\item Déterminer la forme factorisée de $f$.
	\item 
	\begin{enumerate}
		\item L’utilisateur a choisi \cpy{0} pour \cpy{y} et la fonction en \calgpython{} a renvoyé le booléen \cpy{True}. Quelle(s) valeur(s) de \cpy{x} l’utilisateur a-t-il pu choisir ? Justifier.
		\item L’utilisateur a choisi \cpy{-2} pour \cpy{y} et la fonction en \calgpython{} a renvoyé le booléen \cpy{True}. Quelle(s) valeur(s) de \cpy{x} l’utilisateur a-t-il pu choisir ? Justifier.
	\end{enumerate}
\end{enumerate}

\newpage

\exotitre{Exercice 1 - Annexe}

\medskip

\begin{center}
	\tunits{1.5}{0.2}
	\tdefgrille{0}{12}{1}{0.5}{0}{100}{5}{2.5}
	\begin{tikzpicture}[x=\xunit cm,y=\yunit cm]
		\tgrilles[line width=0.3pt,lightgray] ;
		\tgrillep[line width=0.6pt,gray!50] ;
		\axestikz* ;
		\foreach \x in {0,1,...,11} \draw[line width=1.25pt] (\x,4pt) -- (\x,-4pt) ;
		\foreach \y in {0,5,...,90} \draw[line width=1.25pt] (4pt,\y) -- (-4pt,\y) ;
		%\draw[line width=1.5pt,red,domain=0:12,samples=250] plot (\x,{0.6*\x*\x-0.62*\x+18.24}) ;
		%\draw[line width=1.5pt,blue,domain=0:12,samples=250] plot (\x,{7*\x}) ;
	\end{tikzpicture}
\end{center}


\end{document}